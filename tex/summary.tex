%# -*- coding: utf-8-unix -*-
%%==================================================
%% conclusion.tex for SJTUThesis
%% Encoding: UTF-8
%%==================================================
\chapter{总结和展望}\label{chap:summary}
\section{全文总结}
得益于容器虚拟化技术轻便的特性,越来越多的用户正逐渐转向容器云的相关服务。但是对于云服务而言,仅仅是容器本身的轻便带来的帮助还是有限的。特别是在负载变化的场景下,用户需要通过调整服务的实例规模来应对负载的变化,保证自身的服务质量;容器云服务提供商需要提高整体资源使用率,从而降低云服务的成本。降低云服务中服务实例的创建和扩展延迟对提升云服务中的服务启动和伸缩速度,从而提高服务应对负载变化的灵活性,对保障服务质量和降低云服务成本有重要意义。

虽然容器本身相比传统的虚拟机而言更加轻量,能在更短时间内创建更多的容器实例,但是目前已有的容器云在服务的创建和伸缩方面,尤其是扩展方面表现仍然不够理想。除此之外,现有的容器集群管理框架中缺少对服务的副本数进行合理的定义,将服务的实例数兼用作服务的副本数,对保障服务的可用性目标造成很大的干扰。因此本文提出了一个基于容器的主动式容器云资源管理模型,根据服务可用性模型在保证服务可用性目标的前提下,通过基于负载预测提前调整和加速容器云中服务的创建和伸缩性能的方法,使得容器云中的服务能更好地应对负载的变化。

本文将资源使用状态作为负载的衡量手段,从CPU、内存和网络带宽等资源类型对服务和集群节点的资源使用状态进行周期性监测。随后根据监测获得的历史数据,我们分别使用基于整合移动平均自回归模型的预测算法和基于长短期记忆模型的预测算法对资源使用状态进行预测。根据历史记录对接下来的资源使用状态进行预测来提前获知接下来的负载变化趋势,从而提前主动做出相应调整,使得后续只需要根据实际的负载状态对服务规模进行轻微调整,从而加强了系统应对负载变化的灵活性。

本文以容器云中服务实例分布的集群节点作为服务的副本构建容器云中服务的可用性模型,以服务实例分布的节点数来表征服务的可用性,通过控制服务在容器云中节点的分布情况来满足服务的可用性要求。本文通过引入副本数,结合资源监测和预测的资源使用状态,在资源供给和优化模块中确认满足实际负载需求所需要的任务实例数。资源管理模块利用Docker \emph{Registry} API获取运行相关服务所需Docker镜像的元信息,并通过引入新的接口来获取容器云中各节点上全部的Docker镜像层间文件缓存状态。最后,负载优化模块根据运行服务所需的Docker镜像信息和容器云上的层级文件缓存状态,基于启发式的多优先级比较规则确认最终的任务实例调度选择。

我们通过实验对本文提出的模型分别就资源使用状态预测和服务的创建与伸缩这两个方面进行验证。实验结果显示基于该主动式模型实现的系统对服务在容器集群中的创建和伸缩性能有显著提高,在满足服务可用性要求的前提下,相比Docker swarm框架大幅降低了服务创建和服务扩展的延迟。

\section{不足与展望}
尽管我们基于本文提出的模型实现的资源管理系统显著加速了容器云中服务的启动和伸缩过程,但是由于实际中云服务的复杂性和容器本身尚未达成一致的商业化标准,该模型和我们基于该模型实现的系统仍然有很多改进的空间:
\begin{enumerate}
\item 系统目前只支持基于整合移动平均自回归平均模型的预测算法实现和基于长短期记忆模型的预测算法实现,为了进一步提高负载预测的精度,可以引入诸如GARCH等其他预测算法和模型,并在此基础之上对这些算法进行优化,提升最终的预测准确度和稳定性。
\item 目前的模型假定集群中所有节点的可用性都是一样的,而在现实中集群内各节点的可用性指标必然不是全部一致的。因此在未来,我们需要针对由不同可用性指标的节点组成的容器云提出相应的可用性计算模型,从而得到在该场景下服务满足自身可用性目标所需要的副本数目。
\item 目前模型以\emph{REST API}的方式从Docker \emph{registry}服务获取Docker镜像中相应层级文件的数字摘要,从而计算Docker镜像和容器集群中节点上缓存的一致性。然而这些层级文件的数字摘要是基于Docker \emph{registry}服务上压缩过的层级文件生成的。这就使得相比基于未压缩的层级文件生成的数字摘要而言,目前不同层级文件的数字摘要之间出现冲突的可能性更大。因此,我们需要针对Docker \emph{registry}服务设计一个新的\emph{REST API}用来获取基于未压缩的层级文件生成的数字摘要。
\item 我们当前的系统是基于Docker \emph{swarm}实现的,后续可以将该模型应用到诸如Kubernetes和Mesos等其他容器集群管理框架中,构建相应的主动式容器云管理系统。
\end{enumerate}
