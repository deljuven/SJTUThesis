%# -*- coding: utf-8-unix -*-
%%==================================================
%% conclusion.tex for SJTUThesis
%% Encoding: UTF-8
%%==================================================
\chapter{总结和展望}\label{chap:summary}
Hastening startup and spread of services can enhance the flexibility of handling changing loads for the services. Containers are inherently lightweight, whereas the performance of the default scheduler is not good enough for service creation and propagation in Docker swarm. This paper presents the ABP scheduler for expediting the process of creating services and scaling out services to cope with sudden load changes. The results of experiments indicate that the ABP scheduler contributes a remarkable improvement in service creation and distribution in Docker swarm mode.

While there has been significant work done in speeding up service startup and spread in Docker swarm, there are still a lot of rooms to improve the scheduler in the future. The ABP scheduler retrieves layer digests from Docker registry via \emph{REST API} to implement the calculation of coincidence of shared layers from different images. However, the digests are generated from compressed layers on the registry. The possibility of collision between these digests is higher than digests generated from decompressed layers. The development of a new API for Docker registry to generate the digests from decompressed layers will make the scheduling algorithm more accurate. Besides, the ABP scheduler stores the swarm states and other meta information in memory on the running manger node. The scheduler will lose all stored states if leadership changes among manger nodes in the swarm. We could leverage Raft consensus in Docker swarm to share the information among manager nodes to improve the robustness of the ABP scheduler. What's more, the current scheduling algorithm only works for homogeneous swarms, which are built with nodes having same valid probability. A support for heterogeneous swarms which are consisted of nodes having different valid probabilities will make the ABP scheduler applicable in a generic way.
